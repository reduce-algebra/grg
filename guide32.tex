%==========================================================================%
%  GRG 3.2 Reference Guide                  (C) 1988-97 Vadim V. Zhytnikov %
%==========================================================================%
%  This document requires LaTeX 2e. Run LaTeX once:                        %
%                                                                          %
%     latex guide32                                                        %
%                                                                          %
%==========================================================================%

\documentclass[twocolumn]{article}
\addtolength{\voffset}{-10mm}
\addtolength{\textheight}{28mm}
\addtolength{\hoffset}{-8mm}
\addtolength{\textwidth}{10mm}

\usepackage{indentfirst}

%%% This is for CM fonts
\newcommand{\grgtt}{\ttfamily}
\renewcommand{\ttdefault}{cmtt}
\newcommand{\shadedbox}[1]{\fbox{#1}}
\fboxsep=1pt
%%%

%%% Page layout ...
\parindent=0mm
\parskip=2mm
\vfuzz=3pt
%%%

%%% My own \tt font ...
\makeatletter
\def\verbatim@font{\grgtt}
\makeatother
\renewcommand{\tt}{\grgtt}
%%%

%%% Special symbols ...
\def\^{{\tt \char'136}}                     %%%  \^   is  ^
\def\_{{\tt \char'137}}                     %%%  \_   is  _
\newcommand{\w}{{\tt \char'057 \char'134}}  %%%  \w   is  /\
\newcommand{\bs}{{\tt \char'134}}           %%%  \bs  is  \
\newcommand{\ul}{{\tt \char'137}}           %%%  \ul  is  _
\newcommand{\dd}{{\tt \char'043}}           %%%  \dd  is  #
\newcommand{\cc}{{\tt \char'176}}           %%%  \cc  is  ~
\newcommand{\ip}{{\tt \char'137 \char'174}} %%%  \ip  is  _|
\newcommand{\ii}{{\tt \char'174}}           %%%  \ii  is  |
%%%

%%% \grg GRG logo ...
%\newcommand{\grglogofont}{\bfseries}
%\newcommand{\grg}{{\grglogofont GRG}}
\newcommand{\grg}{GRG}

%%% \comm{...} in-line command in the box
\newcommand{\comm}[1]{\shadedbox{\tt#1}}
%%% \command{...} commands in (shaded) box
\newcommand{\command}[1]{\vspace*{1mm}\hfil\break\hspace*{5mm}
\shadedbox{\begin{tabular}{l}
\tt#1 \end{tabular}}\vspace*{0.7mm}\newline}
\newcommand{\longcommand}[1]{\vspace*{1mm}\hfil\break
\shadedbox{\begin{tabular}{l}
\tt#1 \end{tabular}}\vspace*{0.7mm}\newline}

%%% \parm{...} is \itshape for parameters
\newcommand{\parm}[1]{{\slshape\sffamily#1}}
%%% \opt{...} optional
\newcommand{\opt}[1]{{\rm [}#1{\rm ]}}
%%% \rpt{...} repeat
\newcommand{\rpt}[1]{{#1}\,\,{\rm [}{\tt,}{#1}{\tiny\dots}{\rm ]}}

%%% Headings style ...
%\usepackage{fancyheadings}
%%% We just inserat the fancyheadings.sty here literally ...
\makeatletter
% fancyheadings.sty version 1.7
% Fancy headers and footers.
% Piet van Oostrum, Dept of Computer Science, University of Utrecht
% Padualaan 14, P.O. Box 80.089, 3508 TB Utrecht, The Netherlands
% Telephone: +31-30-531806. piet@cs.ruu.nl (mcvax!sun4nl!ruuinf!piet)
% Sep 16, 1994
% version 1.4: Correction for use with \reversemargin
% Sep 29, 1994:
% version 1.5: Added the \iftopfloat, \ifbotfloat and \iffloatpage commands
% Oct 4, 1994:
% version 1.6: Reset single spacing in headers/footers for use with
% setspace.sty or doublespace.sty
% Oct 4, 1994:
% version 1.7: changed \let\@mkboth\markboth to
% \def\@mkboth{\protect\markboth} to make it more robust

\def\lhead{\@ifnextchar[{\@xlhead}{\@ylhead}}
\def\@xlhead[#1]#2{\gdef\@elhead{#1}\gdef\@olhead{#2}}
\def\@ylhead#1{\gdef\@elhead{#1}\gdef\@olhead{#1}}

\def\chead{\@ifnextchar[{\@xchead}{\@ychead}}
\def\@xchead[#1]#2{\gdef\@echead{#1}\gdef\@ochead{#2}}
\def\@ychead#1{\gdef\@echead{#1}\gdef\@ochead{#1}}

\def\rhead{\@ifnextchar[{\@xrhead}{\@yrhead}}
\def\@xrhead[#1]#2{\gdef\@erhead{#1}\gdef\@orhead{#2}}
\def\@yrhead#1{\gdef\@erhead{#1}\gdef\@orhead{#1}}

\def\lfoot{\@ifnextchar[{\@xlfoot}{\@ylfoot}}
\def\@xlfoot[#1]#2{\gdef\@elfoot{#1}\gdef\@olfoot{#2}}
\def\@ylfoot#1{\gdef\@elfoot{#1}\gdef\@olfoot{#1}}

\def\cfoot{\@ifnextchar[{\@xcfoot}{\@ycfoot}}
\def\@xcfoot[#1]#2{\gdef\@ecfoot{#1}\gdef\@ocfoot{#2}}
\def\@ycfoot#1{\gdef\@ecfoot{#1}\gdef\@ocfoot{#1}}

\def\rfoot{\@ifnextchar[{\@xrfoot}{\@yrfoot}}
\def\@xrfoot[#1]#2{\gdef\@erfoot{#1}\gdef\@orfoot{#2}}
\def\@yrfoot#1{\gdef\@erfoot{#1}\gdef\@orfoot{#1}}

\newdimen\headrulewidth
\newdimen\footrulewidth
\newdimen\plainheadrulewidth
\newdimen\plainfootrulewidth
\newdimen\headwidth
\newif\if@fancyplain \@fancyplainfalse
\def\fancyplain#1#2{\if@fancyplain#1\else#2\fi}

% Command to reset various things in the headers:
% a.o.  single spacing (taken from setspace.sty)
% and the catcode of ^^M (so that epsf files in the header work if a
% verbatim crosses a page boundary)
\def\fancy@reset{\restorecr
 \def\baselinestretch{1}%
 \ifx\undefined\@newbaseline% NFSS not present; 2.09 or 2e
  \ifx\@currsize\normalsize\@normalsize\else\@currsize\fi%
 \else% NFSS (2.09) present
  \@newbaseline%
 \fi}

% Initialization of the head and foot text.

\headrulewidth 0.4pt
\footrulewidth\z@
\plainheadrulewidth\z@
\plainfootrulewidth\z@

\lhead[\fancyplain{}{\sl\rightmark}]{\fancyplain{}{\sl\leftmark}}
%  i.e. empty on ``plain'' pages \rightmark on even, \leftmark on odd pages
\chead{}
\rhead[\fancyplain{}{\sl\leftmark}]{\fancyplain{}{\sl\rightmark}}
%  i.e. empty on ``plain'' pages \leftmark on even, \rightmark on odd pages
\lfoot{}
\cfoot{\rm\thepage} % page number
\rfoot{}

% Put together a header or footer given the left, center and
% right text, fillers at left and right and a rule.
% The \lap commands put the text into an hbox of zero size,
% so overlapping text does not generate an errormessage.

\def\@fancyhead#1#2#3#4#5{#1\hbox to\headwidth{\fancy@reset\vbox{\hbox
{\rlap{\parbox[b]{\headwidth}{\raggedright#2\strut}}\hfill
\parbox[b]{\headwidth}{\centering#3\strut}\hfill
\llap{\parbox[b]{\headwidth}{\raggedleft#4\strut}}}\headrule}}#5}


\def\@fancyfoot#1#2#3#4#5{#1\hbox to\headwidth{\fancy@reset\vbox{\footrule
\hbox{\rlap{\parbox[t]{\headwidth}{\raggedright#2\strut}}\hfill
\parbox[t]{\headwidth}{\centering#3\strut}\hfill
\llap{\parbox[t]{\headwidth}{\raggedleft#4\strut}}}}}#5}

\def\headrule{{\if@fancyplain\headrulewidth\plainheadrulewidth\fi
\hrule\@height\headrulewidth\@width\headwidth \vskip-\headrulewidth}}

\def\footrule{{\if@fancyplain\footrulewidth\plainfootrulewidth\fi
\vskip-0.3\normalbaselineskip\vskip-\footrulewidth
\hrule\@width\headwidth\@height\footrulewidth\vskip0.3\normalbaselineskip}}

\def\ps@fancy{
\def\@mkboth{\protect\markboth}
\@ifundefined{chapter}{\def\sectionmark##1{\markboth
{\uppercase{\ifnum \c@secnumdepth>\z@
 \thesection\hskip 1em\relax \fi ##1}}{}}
\def\subsectionmark##1{\markright {\ifnum \c@secnumdepth >\@ne
 \thesubsection\hskip 1em\relax \fi ##1}}}
{\def\chaptermark##1{\markboth {\uppercase{\ifnum \c@secnumdepth>\m@ne
 \@chapapp\ \thechapter. \ \fi ##1}}{}}
\def\sectionmark##1{\markright{\uppercase{\ifnum \c@secnumdepth >\z@
 \thesection. \ \fi ##1}}}}
\ps@@fancy
\global\let\ps@fancy\ps@@fancy
\headwidth\textwidth}
\def\ps@fancyplain{\ps@fancy \let\ps@plain\ps@plain@fancy}
\def\ps@plain@fancy{\@fancyplaintrue\ps@@fancy}
\def\ps@@fancy{
\def\@oddhead{\@fancyhead\@lodd\@olhead\@ochead\@orhead\@rodd}
\def\@oddfoot{\@fancyfoot\@lodd\@olfoot\@ocfoot\@orfoot\@rodd}
\def\@evenhead{\@fancyhead\@rodd\@elhead\@echead\@erhead\@lodd}
\def\@evenfoot{\@fancyfoot\@rodd\@elfoot\@ecfoot\@erfoot\@lodd}
}
\def\@lodd{\if@reversemargin\hss\else\relax\fi}
\def\@rodd{\if@reversemargin\relax\else\hss\fi}

\let\latex@makecol\@makecol
\def\@makecol{\let\topfloat\@toplist\let\botfloat\@botlist\latex@makecol}
\def\iftopfloat#1#2{\ifx\topfloat\empty #2\else #1\fi}
\def\ifbotfloat#1#2{\ifx\botfloat\empty #2\else #1\fi}
\def\iffloatpage#1#2{\if@fcolmade #1\else #2\fi}
\makeatother
%%%
\pagestyle{fancy}

\headrulewidth=0.1mm
\footrulewidth=0.1mm

\lhead{\bf\slshape GRG 3.2 Reference Guide}
\chead{}
\rhead{\bf\thepage}

\lfoot{}
\cfoot{}
\rfoot{}
%%%

%%% Sections ...
\renewcommand{\thesection}{\hspace*{-5mm}}
\renewcommand{\thesubsection}
   {{\sf\slshape\arabic{subsection}.}\hspace*{-3mm}}


\begin{document}

%\title{\LARGE\bf \grg\ 3.2 Reference Guide\vspace*{-8mm}}
%\date{}
%\maketitle

%\raggedright
\footnotesize


\section{\LARGE\sf\slshape Commands}
\chead{\slshape Commands}

\tabcolsep=0.5mm

\grg\ commands are not case sensitive, i.e. they can be
typed in lower, upper or mixed case.  Optional parts of the
commands are enclosed in square brackets \opt{\parm{x}}
and construction \rpt{\parm{x}} stands for {\tt \parm{x}} or
{\tt \parm{x},\,\parm{x}} or {\tt \parm{x},\,\parm{x},\,\parm{x}} etc.


\subsection{\sf\slshape Session Control Commands}

The command \comm{Quit;} terminates both \grg\ and {\sc Reduce}
sessions. The command \comm{Stop;} terminates \grg\ task and
brings the session control menu.

Batch file execution:
\command{\opt{Input} "\parm{file}";}
The batch file execution can be suspended by the command
\comm{Pause;} and resumed by the command \comm{Next;}.

The command \comm{Output "\parm{file}";}\vspace*{0.4mm} redirects
all \grg\ output into the \parm{file}.
The command \comm{EndO;} or \comm{End of Output;} closes
the \parm{file} and restores standard output.


\subsection{\sf\slshape Operating System Commands}

The command \comm{System;} suspend \grg\ session
and passes control to the operating system command level.
The command \comm{System "\parm{command}";}
executes single operating system \parm{command}.


\subsection{\sf\slshape Comments}\vspace{-5mm}

\command{Comment \parm{any text};\\\tt
\parm{any command} \% \parm{any text};\\\tt
\% \parm{any text};}


\subsection{\sf\slshape Switches Control Commands}

The commands
\command{On \rpt{\parm{switch}}; \\\tt
         Off \rpt{\parm{switch}};}
change the \parm{switch} position and the command
\command{\opt{Show} Switch \parm{switch};\\\tt Show \parm{switch};}
prints current \parm{switch} status.


\subsection{\sf\slshape Info Commands}

Time and garbage collection time commands:
\command{\opt{Show} Time;\\\tt
\opt{Show} GC Time;}
The timer can be set to zero by the command \comm{Zero Time;}.

The command
\command{\opt{Show} Status;}
print information about the current system directory,
type of the metric, frame and basis.

The command \comm{Show *;} prints the list of all built-in
objects. The command \comm{Show a*;} prints the list of the
built-in objects whose names begins with the character {\tt a}.
Finally the command
\command{Show \parm{object};}
prints detailed information about the \parm{object} including its
name, symbol, indices, symmetries, type of the component,
current state and ways of calculation.

The command \comm{Show All;} prints a list of objects whose
values are currently known.


\subsection{\sf\slshape Declarations}

The dimension and signature declaration
\command{Dimension \parm{dim} with \opt{Signature} (\rpt{\parm{pm}});}
where \parm{pm} is {\tt +} or {\tt -}.

The coordinates and constants declarations
\command{Coordinates \rpt{\parm{x}};\\\tt
         Constants \rpt{\parm{c}};}

The functions and generic function declarations
\command{Functions \rpt{\parm{f}\,\,\opt{{\upshape (}\rpt{\parm{x}}{\upshape )}}};\\\tt
Generic Functions \rpt{\parm{f}\,\,{\upshape (}\rpt{\parm{x}}{\upshape )}};}

Function properties declaration
\command{Symmetric \rpt{\parm{f}};\\\tt
Antisymmetric \rpt{\parm{f}};\\\tt
Odd \rpt{\parm{f}};\\\tt
Even \rpt{\parm{f}}; }

The command \comm{Affine Parameter \parm{s};} declares
the affine parameter.


\subsection{\sf\slshape New Object Declaration}

The following equivalent declarations
\command{New Object \parm{ID}\,\opt{\parm{ilst}}\,\opt{is \parm{ctype}}\,\opt{with \opt{Symmetries}\,\parm{slst}};\\\tt
Object \parm{ID}\,\opt{\parm{ilst}}\,\opt{is \parm{ctype}}\,\opt{with \opt{Symmetries}\,\parm{slst}};\\\tt
New \parm{ID}\,\opt{\parm{ilst}}\,\opt{is \parm{ctype}}\,\opt{with \opt{Symmetries}\,\parm{slst}}; }
introduce new user-defined object, equation
\command{New Equation \parm{ID}\,\opt{\parm{ilst}}\,\opt{is \parm{ctype}}\,\opt{with \opt{Symmetries}\,\parm{slst}};\\\tt
Equation \parm{ID}\,\opt{\parm{ilst}}\,\opt{is \parm{ctype}}\,\opt{with \opt{Symmetries}\,\parm{slst}}; }
or connection 1-form
\command{New Connection \parm{ID}\,\opt{\parm{ilst}}\,\opt{is 1-form};\\\tt
Connection \parm{ID}\,\opt{\parm{ilst}}\,\opt{is 1-form}; }

Here \parm{ilst} is the index type list
\comm{\rpt{\parm{ipos}\ \parm{itype}}}
where \parm{ipos} is one of the markers denoting the
index position
\command{{\tt '}\rm\ \ upper frame
\\{\tt .}\rm\ \ lower frame
\\{\tt \^}\rm\ \ upper holonomic
\\{\tt \ul}\rm\ \ lower holonomic }
and \parm{itype} determines index type. For example:
holonomic or frame indices {\tt a b c}, enumerating indices
{\tt i3 i15 idim}, spinor {\tt A PQ MNL} and conjugated spinor
indices {\tt A\cc\ PQ\cc\ MNL\cc}.

The \parm{ctype} defines the type of the component:
\command{Scalar \opt{Density \parm{dens}}\\\tt
\parm{n}-form \opt{Density \parm{dens}}\\\tt
Vector \opt{Density \parm{dens}}}
The \parm{dens} defines pseudo-scalar and density
properties of the object with respect to
coordinate and frame transformations:
\command{\opt{sgnL}\opt{*sgnD}\opt{*L\^\parm{n}}\opt{*D\^\parm{m}}}
where \comm{D} and \comm{L} is the coordinate and frame
transformation determinants respectively.

The symmetry specification \parm{slst} is a list \rpt{\parm{slst1}}.
Each \parm{slst1} is {\tt \parm{sym}(\rpt{\parm{slst2}})}
where \parm{sym} is: \comm{a} for antisymmetry, \comm{t} for symmetry,
\comm{c} for cyclic symmetry and \comm{h} for Hermitian symmetry.
The \parm{slst2} is either index number, or list of index numbers
or once again another symmetry specification \parm{slst1}.

The command \comm{Forget \parm{object};} removes the
user-defined \parm{object}.


\subsection{\sf\slshape Assignment}

The command
\command{\opt{\parm{Name}}\,\rpt{\parm{ID}\,\opt{{\upshape(}\rpt{\parm{i}}{\upshape)}}=\parm{expr}};}
assigns the value to the component(s) of the object \parm{Name}
having the symbol \parm{ID}.


\subsection{\sf\slshape Object Calculation}

The command for calculating the value of an \parm{object}
using built-in \parm{way} (formula):
\command{Find \rpt{\parm{object}}\,\opt{\parm{way}};}
Here \parm{object} is either the name or the symbol of
the built-in object. The \parm{way} is either the name of the
way or any object which is present at the right-hand side of
the formula.

The command
\command{Null Metric;}
makes the metric to be the \emph{standard null metric}.

The command
re-simplifies the \parm{object}.
The command
\command{Erase \parm{object};}
removes the value of the \parm{object}
and makes it indefinite once again. The command
\command{Zero \parm{object};}
assigns zero value to the \parm{object}.
The command
\command{Normalize \parm{equation};}
replaces equation $l=r$ by $l-r=0$.


\subsection{\sf\slshape Object Printing}

The command
\command{Write \rpt{\parm{object}}\,\,\opt{to "\parm{file}"};}
prints the value of the \parm{object} (to the \parm{file} if present).

The command
\command{Write \opt{to "\parm{file}"};}
redirects all output into the \parm{file}.
The command \comm{EndW;} or \comm{End of Write;}
closes the \parm{file} and restores standard output.

The symbol {\tt >} can be used instead of {\tt to} in these commands.

%\newpage

The following commands print the line-element:
\command{ds2;\\\tt
Line-Element;}


\subsection{\sf\slshape Expression Printing}

The following commands evaluate expression \parm{expr}
and print its value:
\command{\opt{Print} \parm{expr} \opt{For \parm{iter}};\\\tt
For \parm{iter} Print \parm{expr};}
The parameter \parm{iter} determines that the \parm{expr}
must be evaluated for several values of some variable.
The \parm{iter} has the form:
\command{\rpt{\parm{it}\,\opt{=\opt{\parm{lo}{\upshape..}}\parm{up}}}}
The separator {\tt ,} can be replaced by one of the relational
operators {\tt <\ \ >\ \ <=\ \ >=}. In general \parm{it} runs
from \parm{lo} (or from 0 if \parm{lo} is omitted) to \parm{up}.
If both \parm{lo} and \parm{up} are omitted then range of the
symbol \parm{it} is determined by its form. For example:
{\tt a p ijk} run from 0 to $d-1$ ($d$ is the dimension),
{\tt a5 ij5} run from 0 to 5, {\tt a13 ij13} run from 1 to 3,
{\tt A} runs from 0 to 1, {\tt AB} runs from 0 to 2,
{\tt ABC} runs from 0 to 3 etc.


\subsection{\sf\slshape Output Control}

The following commands are identical to
\command{Factor \rpt{\parm{expr}};\\\tt
RemFac \rpt{\parm{expr}};\\\tt
Order \rpt{\parm{expr}};}
similar {\sc Reduce} commands.
The command \comm{Line-Length \parm{n};} sets new output
line width.


\subsection{\sf\slshape Substitutions}

The substitution commands are similar to corresponding
{\sc Reduce} instructions
\command{\opt{For All \rpt{\parm{x}}\,\opt{Such That \parm{cond}}} Let \rpt{\parm{sub}};\\\tt
\opt{For All \rpt{\parm{x}}\,\opt{Such That \parm{cond}}} Match \rpt{\parm{sub}};\\\tt
\opt{For All \rpt{\parm{x}}\,\opt{Such That \parm{cond}}} Clear \rpt{\parm{sub}}; }
where \parm{sub} is either relation {\tt \parm{l}\,=\,\parm{r}} as in
{\sc Reduce} or component of the solution {\tt Sol(\parm{n})}.


\subsection{\sf\slshape Basis Mode Switching Commands}

The command
\command{Anholonomic;}
switch \grg\ to the anholonomic basis mode and the command
\command{Holonomic;}
switches back to the default holonomic mode.


\subsection{\sf\slshape Saving and Restoring the Data}

The command
\command{Unload \rpt{\parm{object}} to "\parm{file}";}
saves the value of the \parm{object} into the \parm{file}.

The command
\command{Unload to "\parm{file}";}
must be followed by the sequence of the commands
\command{Unload \parm{object};}
or comments. The sequence must be terminated
by the command \comm{EndU;} or \comm{End of Unload;}.

The symbol {\tt >} can be used instead of {\tt to}.

The data saved by {\tt Unload} can be restored by the command
\command{Load "\parm{file}";}

The command
\command{\opt{Show} File "\parm{file}";\\\tt Show "\parm{file}";}
lists the objects saved into the \parm{file}.


\subsection{\sf\slshape Algebraic Classification}

The command
\command{Classify \parm{object};}
performs algebraic classification of the \parm{object}.
\grg\ has built-in algorithms for the algebraic
classification of the following irreducible spinors:
$X_{A\dot{B}}$, $X_{AB}$, $X_{AB\dot{C}\dot{D}}$, $X_{ABCD}$.


\subsection{\sf\slshape Coordinate Transformations}

The coordinate transformation command:
\longcommand{New Coordinates \rpt{\parm{new}} with \rpt{\parm{old}=\parm{expr}};}


\subsection{\sf\slshape Frame Transformations}

Frame rotation command
\command{\opt{Make} Rotation \opt{\parm{matrix}};}
The \parm{matrix} must be frame rotation, i.e. the metric must
remain unchanged under the transformation. The \parm{matrix}
has the following form
\command{{\upshape (}\rpt{{\upshape (}\rpt{\parm{expr}}{\upshape )}}{\upshape )}}
If \parm{matrix} is omitted then the rotation is taken from
the object {\tt Frame Transformation}.

The command
\command{Change Metric \opt{\parm{matrix}};}
is similar to the previous one but the \parm{matrix}
is not necessary the rotation but any nonsingular matrix.

The spinorial transformation command:
\command{\opt{Make} Spinorial Rotation \opt{\parm{matrix}};}
The \parm{matrix} must be SL(2,C) matrix.
If the parameter \parm{matrix} is omitted
then the matrix must be defined by the value of the
object {\tt Spinorial Transformation}.

The command
\command{Hold \parm{object};}
makes \grg\ to keep the \parm{object} unchanged under
the frame transformation. The command
\command{Release \parm{object};}
removes the action of the {\tt Hold} command.


\subsection{\sf\slshape Solving Equations}

The algebraic equation solving command has two forms
\command{Solve \parm{equation} for \rpt{\parm{x}};\\\tt
Solve \rpt{\parm{l}=\parm{r}}\,\,for \rpt{\parm{x}};}
where \parm{equation} is any built-in or user-defined
equation. The solutions are stored into the special
built-in object {\tt Solutions}.

The command
\command{\tt Inverse \parm{f},\,\parm{h};}
declares the functions \parm{f} and \parm{h} to be inverse
to each other.


\subsection{\sf\slshape Loading Package}

\command{\opt{Load} Package \parm{package};\\\tt
Load \parm{package};}


\section{\LARGE\sf\slshape Switches}\vspace*{-2mm}
\chead{\slshape Commands and Switches}

Switches in \grg\ are case insensitive.

\tabcolsep=1.5mm

\begin{tabular}{|c|c|l|}
\hline
\tt  AEVAL          & Off & Use aeval() instead of reval().          \\
\tt  WRS            & On  & Re-simplify expr. before printing.     \\
\tt  WMATR          & Off & Write 2-index objects in matrix form.   \\
\tt  TORSION        & Off & Torsion.                                 \\
\tt  NONMETR        & Off & Nonmetricity.                            \\
\tt  UNLCORD        & On  & Save coordinates in {\tt Unload}.       \\
\tt  AUTO           & On  & Automatic data calculation in expr.    \\
\tt  TRACE          & On  & Trace the calculation process.          \\
\tt  SHOWCOMMANDS   & Off & Show compound command expansion.       \\
\tt  EXPANDSYM      & Off & Allow {\tt Sy Asy Cy}in expr.        \\
\tt  DFPCOMMUTE     & On  & Commutativity of {\tt DFP}.              \\
\tt  NONMIN         & Off & Nonmin. interaction for scalar field.    \\
\tt  NOFREEVARS     & Off & Prohibit free variables in {\tt Print}. \\
\tt  CCONST         & Off & Include cosm. constant in equations.    \\
\tt  FULL           & Off & Number of components in {\tt Metric Eq}. \\
\tt  LATEX          & Off &  \LaTeX\ output mode.                    \\
\tt  GRG            & Off &  \grg\ output mode.                      \\
\tt  REDUCE         & Off &  {\sc Reduce} output mode.               \\
\tt  MAPLE          & Off &  {\sc Maple} output mode.                \\
\tt  MATH           & Off &  {\sc Mathematica} output mode.          \\
\tt  MACSYMA        & Off &  {\sc Macsyma} output mode.              \\
\tt  DFINDEXED      & Off & Print {\tt DF} in index notation.       \\
\tt  BATCH          & Off & Batch mode.                              \\
\tt  HOLONOMIC      & On  & Keep frame holonomic.   \\
\tt  SHOWEXPR       & Off & Print expressions during algebraic      \\
\tt          	    &     & classification.                          \\
\hline
\end{tabular}

\newpage



\section{\LARGE\sf\slshape Synonymy}
\chead{\slshape Synonymy}

This is default \grg\ synonymy list.
The symbols in each line are equivalent in all
\grg\ commands and in the built-in object names.
The case does not matter. So {\tt Affine} is
equivalent to {\tt affine}, {\tt Aff}, {\tt aff}
and so on.

\begin{verbatim}
   Affine Aff
   Anholonomic Nonholonomic AMode ABasis
   Antisymmetric Asy
   Change Transform
   Classify Class
   Components Comp
   Connection Con
   Constants Const Constant
   Coordinates Cord
   Curvature Cur
   Dimension Dim
   Dotted Do
   Equation Equations Eq
   Erase Delete Del
   Evaluate Eval Simplify
   Find F Calculate Calc
   Form Forms
   Functions Fun Function
   Generic Gen
   Gravitational Gravity Gravitation Grav
   Holonomic HMode HBasis
   Inverse Inv
   Load Restore
   Next N
   Normalize Normal
   Object Obj
   Output Out
   Parameter Par
   Rotation Rot
   Scalar Scal
   Show ?
   Signature Sig
   Solutions Solution Sol
   Spinor Spin Spinorial Sp
   standardlisp lisp
   Switch Sw
   Symmetries Sym Symmetric
   Tensor Tensors Tens
   Torsion Tors
   Transformation Trans
   Undotted Un
   Unload Save
   Vector Vec
   Write W
   Zero Nullify
\end{verbatim}

\newpage


\section{\LARGE\sf\slshape Expressions}
\chead{\slshape Expressions}

\subsection{\sf\slshape Operations and Operators}

Notation:
$e$ is any expression,
$a$ is any scalar valued (algebraic) expressions,
$v$ is any vector valued expression,
$x$ is a coordinate,
$o$ is any 1-form valued expression,
$\omega$ is any form valued expression.

\begin{tabular}{|c|c|c|}
\hline
{\tt [$v_1$,$v_2$]} & Vector bracket          &                             \\
\hline
{\tt @} $x$         & Holonomic vector $\partial_x$ &                       \\
\cline{1-2}
{\tt d} $a$         & Exterior differential   &                             \\
{\tt d} $\omega$    &                         &
          {\tt d} \cc$a$ $\Leftrightarrow$ {\tt (d(}\cc$a${\tt))} \\
\cline{1-2}
{\tt \dd} $a$       & Dualization             &                             \\
{\tt \dd} $\omega$  &                         &                             \\
\cline{1-2}
{\tt \cc} $e$       & Complex conjugation     &                             \\
\hline
$a_1${\tt **}$a_2$  & Exponention             &                             \\
$a_1${\tt\^} $a_2$  &                         &                             \\
\hline
$e$\ {\tt /}\ $a$   & Division                &
          $e${\tt /}$a_1${\tt /}$a_2$ $\Leftrightarrow$ {\tt (}$e${\tt /}$a_1${\tt )/}$a_2$  \\
\hline
$a$\ {\tt *}\ $e$   & Multiplication          &                                   \\
\cline{1-2}
$v$\ {\tt |}\ $a$   & Vector acting on scalar & $v$\ii$\omega_1$\w$\omega_2${\tt *}$a$ \\
\cline{1-2}
$v$\ \ip\ $\omega$  & Interior product        & $\Updownarrow$  \\
\cline{1-2}
$v_1$\ {\tt.}\ $v_2$& Scalar product          & $v$\ii{\tt (}$\omega_1$\w{\tt(}$\omega_2${\tt *}$a${\tt ))} \\
$v$\ {\tt.}\ $o$    &                         &                             \\
$o_1$\ {\tt.}\ $o_2$&                         &                             \\
\cline{1-2}
$\omega_1$\ \w\ $\omega_2$ & Exterior product &                             \\
\hline
{\tt +}\ $e$        & Prefix plus             &                             \\
\cline{1-2}
{\tt -}\ $e$        & Prefix minus            &                             \\
\cline{1-2}
$e_1$\ {\tt +}\ $e_2$ & Addition              &                             \\
\cline{1-2}
$e_1$\ {\tt -}\ $e_2$ & Subtraction           &                             \\
\hline
\end{tabular}


\subsection{\sf\slshape Variables and Functions}

Operator listed in the previous section can act on:
(i) integer numbers (e.g. {\tt 0}, {\tt 123}),
(ii) symbols or identifiers (e.g. {\tt I}, {\tt phi}, {\tt RIM0103}),
(iii) functional expressions (e.g. {\tt SIN(x)}, {\tt G(0,1)} etc).

Valid symbol must belong to one of the following types:
\begin{itemize}
\item Coordinate.
\item Declared by user or built-in constant.
\item Function declared with implicit dependence list.
\item Component of an object.
\end{itemize}

Any valid functional expression must belong to one of the following types:
\itemsep=0.5mm
\begin{itemize}
\item User-defined function.
\item Function defined in {\sc Reduce} or in any loaded package.
\item Component of an object in functional notation.
\item Some special \grg\ functional expressions listed below.
\end{itemize}


\subsection{\sf\slshape Object Components}

The components of built-in or user-defined object can be
referred by two methods: using symbols {\tt dim},
{\tt VOL}, {\tt T0}, {\tt RIM0213} etc, or using functional
notation {\tt T(0)}, {\tt RIM(0,2,1,3)}, {\tt OMEGA(i,j)}.
In functional notation the default index type and position
can be changed using the markers: {\tt '} upper frame,
{\tt .} lower frame, {\tt \^} upper holonomic, {\tt \_} lower
holonomic. For example: {\tt RIM('0,.1,\_2,\_3)}.



\subsection{\sf\slshape Built-in Constants}

\begin{tabular}{|l|l|}
\hline
\tt  E I PI INFINITY     & Mathematical constants $e,i,\pi$,$\infty$    \\
\hline
\tt  FAILED              &                                             \\
\hline
\tt  ECONST              & Charge of the electron                      \\
\tt  DMASS               & Dirac field mass                            \\
\tt  SMASS               & Scalar field mass                           \\
\hline
\tt  GCONST              & Gravitational constant                      \\
\tt  CCONST              & Cosmological constants                      \\
\hline
\tt  LC0 LC1 LC2 LC3     & Parameters of the quadratic                 \\
\tt  LC4 LC5 LC6         & gravitational Lagrangian                    \\
\tt  MC1 MC2 MC3         &                                             \\
\hline
\tt  AC0                 & Nonminimal interaction constant             \\
\hline
\end{tabular}


\subsection{\sf\slshape Derivatives}\vspace*{-5mm}

\command{DF(\parm{a},\rpt{\parm{x}\opt{{\upshape ,}\parm{n}}})\\\tt
DFP(\parm{a},\rpt{\parm{x}\opt{{\upshape ,}\parm{n}}})}\vspace*{-1mm}

{\tt DFP} derivatives are valid only after {\tt Generic Function}
declaration.

\subsection{\sf\slshape Complex Conjugation}

These constructions are shortcuts for standard complex conjugation
operations:
\command{%
\tt $e$ + \cc\cc\ $=$\ $e$ + \cc$e$ \\
\tt $e$ - \cc\cc\ $=$\ $e$ - \cc$e$ \\
\tt Re($e$)\ $=$\ ($e$ + \cc$e$)/2 \\
\tt Im($e$)\ $=$\ I*(-$e$ + \cc$e$)/2}


\subsection{\sf\slshape Parts of Equations and Solutions}

The functional expressions
\command{LHS(\parm{eqcomp})\\\tt
RHS(\parm{eqcomp})}
give access to the left-hand and right-hand side of an
equation respectively. They also provide access to the \parm{n}'th
solution if \parm{eqcomp} is \comm{Sol(\parm{n})}.


\subsection{\sf\slshape Sums and Products}\vspace*{-5mm}

\command{Sum(\parm{iter},\parm{e})\\\tt
Prod(\parm{iter},\parm{e})}
The \parm{iter} specification is
completely the same as in the {\tt Print For} command.


\subsection{\sf\slshape Lie Derivatives}

The Lie derivative
\command{Lie(\parm{v},\parm{objcomp})}
where \parm{objcomp} is the component of an object in
functional notation.


\subsection{\sf\slshape Covariant Derivatives and Differentials}

The covariant differential
\command{Dc(\parm{objcomp}\opt{{\upshape\tt ,}\rpt{\parm{conn}}})}
and covariant derivative
\command{Dfc(\parm{v},\parm{objcomp}\opt{{\upshape\tt ,}\rpt{\parm{conn}}})}
Here \parm{objcomp} is an object component in functional notation
and \parm{conn} is the symbol(s) of alternative connection 1-form(s).


\subsection{\sf\slshape Symmetrization}

The functional expressions
\command{%
Asy(\rpt{\parm{i}},\parm{e})\\\tt
Sy(\rpt{\parm{i}},\parm{e})\\\tt
Cy(\rpt{\parm{i}},\parm{e})}
produces antisymmetrization, symmetrization and cyclic symmetrization
of the expression \parm{e} with respect to \parm{i} (without
corresponding $1/n$ or $1/n!$ etc). The switch {\tt EXPANDSYM} must
be on.

\subsection{\sf\slshape Substitutions}

The expression
\command{SUB(\rpt{\parm{sub}},\parm{e})}
is similar to the analogous {\sc Reduce} one with two
generalizations: (i) it applies not only to algebraic
but to form and vector valued expression \parm{e} as well,
(ii) as in {\tt Let} command \parm{sub} can be either
the relation {\tt \parm{l}\,=\,\parm{r}} or solution
{\tt Sub(\parm{n})}.


\subsection{\sf\slshape Conditional Expressions}

The conditional expression
\command{If(\parm{cond},\parm{$e_1$},\parm{$e_2$})}
chooses $e_1$ or $e_2$ depending on the value of the
boolean expression \parm{cond}.

Boolean expression appears in (i) the conditional expression
{\tt If}, (ii) in {\tt For all Such That} substitutions.
Any nonzero expression is considered as {\bf true} and
vanishing expression as {\bf false}. Boolean expressions
may contain the following usual relations and logical
operations: {\tt < > <= >= = |= not and or}. They also may
contain the predicates

\begin{tabular}{|l|l|}
\hline
\tt OBJECT(\parm{obj}) & Is \parm{obj} an object or not          \\
\hline
\tt ON(\parm{switch})      & Test position of the \parm{switch}      \\
\tt OFF(\parm{switch})    &                                            \\
\hline
\tt ZERO(\parm{object})    & Is the value of the \parm{object} zero or not \\
\hline
\tt HASVALUE(\parm{object}) & Whether the \parm{object} has any value or not \\
\hline
\tt NULLM(\parm{object}) & Is the \parm{object} the standard null metric \\
\hline
\end{tabular}

The expression \comm{ERROR("\parm{message}")} causes an error
with the \comm{"\parm{message}"}. It can be used together with
conditional expressions to test any required conditions during
the batch file execution.



\newpage

\section{\LARGE\sf\slshape Macro Objects}
\chead{\slshape Objects}

Macro objects can be used in expression, in {\tt Write} and
{\tt Show} commands but not in {\tt Find}. The indices are
specified as in the {\tt New Object} declaration.

\subsection{\sf\slshape Dimension and Signature}

\begin{tabular}{|l|l|}
\hline
\tt  dim       &  Dimension $d$ \\
\hline
\tt  sdiag.idim & {\tt sdiag(\parm{n})} is the $n$'th element of the \\
                &  signature diag($-1,+1$\dots) \\
\hline
\tt  sign      &  Product of the signature specification \\
\tt  sgnt      &  elements $\prod_{n=0}^{d-1}\mbox{\tt sdiag(}n\mbox{\tt)}$ \\[1mm]
\hline
\tt  mpsgn     &  {\tt sdiag(0)}  \\
\tt  pmsgn     &  {\tt -sdiag(0)}   \\
\hline
\end{tabular}

\subsection{\sf\slshape Metric and Frame}

\begin{tabular}{|l|l|}
\hline
\tt  x\^m        &  $m$'th coordinate                   \\
\tt  X\^m        &                     \\
\hline
\tt  h'a\_m    &  Frame coefficients         \\
\tt  hi.a\^m   &                    \\
\hline
\tt  g\_m\_n    & Holonomic metric      \\
\tt  gi\^m\^n   &                   \\
\hline
\end{tabular}

\subsection{\sf\slshape Delta and Epsilon Symbols}

\begin{tabular}{|l|l|}
\hline
\tt  del'a.b       &  Delta symbols   \\
\tt  delh\^m\_n    &                  \\
\hline
\tt  eps.a.b.c.d   &  Totally antisymmetric symbols \\
\tt  epsi'a'b'c'd  &  (number of indices depend on $d$)  \\
\tt  epsh\_m\_n\_p\_q  &                     \\
\tt  epsih\^m\^n\^p\^q &                     \\
\hline
\end{tabular}

\subsection{\sf\slshape Spinors}

\begin{tabular}{|l|l|}
\hline
\tt  DEL'A.B      & Delta symbol          \\
\hline
\tt  EPS.A.B      & Spinorial metric      \\
\tt  EPSI'A'B     &                       \\
\hline
\tt  sigma'a.A.B\cc   & Sigma matrices      \\
\tt  sigmai.a'A'B\cc  &                    \\
\hline
\tt  cci.i3    & Frame index conjugation in st. null frame \\
	       & {\tt cci(0)=0}\ {\tt cci(1)=1}\ {\tt cci(2)=3}\ {\tt cci(3)=2} \\
\hline
\end{tabular}

\subsection{\sf\slshape Connection Coefficients}

\begin{tabular}{|l|l|}
\hline
\tt  CHR\^m\_n\_p  &  Christoffel symbols $\{{}^\mu_{\nu\pi}\}$ \\
\tt  CHRF\_m\_n\_p &  and $[{}_{\mu},_{\nu\pi}]$  \\
\tt  CHRT\_m       &  Christoffel symbol trace $\{{}^\pi_{\pi\mu}\}$  \\
\hline
\tt  SPCOEF.AB.c     & Spin coefficients $\omega_{AB\,c}$  \\
\hline
\end{tabular}

\subsection{\sf\slshape NP Formalism}

\begin{tabular}{|l|c|}
\hline
\tt  PHINP.AB.CD~ &  $\Phi_{AB\dot{C}\dot{D}}$  \\
\tt  PSINP.ABCD   &  $\Psi_{ABCD}$              \\
\hline
\tt  alphanp      & $\alpha$ \\
\tt  betanp       & $\beta$ \\
\tt  gammanp      & $\gamma$ \\
\tt  epsilonnp    & $\epsilon$ \\
\tt  kappanp      & $\kappa$ \\
\tt  rhonp        & $\rho$ \\
\tt  sigmanp      & $\sigma$ \\
\tt  taunp        & $\tau$ \\
\tt  munp         & $\mu$ \\
\tt  nunp         & $\nu$ \\
\tt  lambdanp     & $\lambda$ \\
\tt  pinp         & $\pi$ \\
\hline
\tt  DD           & $D$ \\
\tt  DT           & $\Delta$ \\
\tt  du           & $\delta$ \\
\tt  dd           & $\overline\delta$ \\
\hline
\end{tabular}



\section{\LARGE\sf\slshape Built-in Objects}

\tabcolsep=1mm

The complete list of built-in objects with names and symbols.
The case of the object names is not important but symbols
are case sensitive. The indices are specified as in the
{\tt New Object} declaration. Some names refer to a set
of objects. For example the name {\tt Spinorial S - forms}
denotes {\tt SU.AB} and {\tt SD.AB~}.

\subsection{\sf\slshape  Metric, Frame, Basis, Volume \dots}
\begin{tabular}{|l|l|}\hline
\tt    Frame                   &\tt   T'a\\
\tt    Vector Frame            &\tt   D.a\\
\hline
\tt    Metric                  &\tt   G.a.b\\
\tt    Inverse Metric          &\tt   GI'a'b\\
\tt    Det of Metric           &\tt   detG\\
\tt    Det of Holonomic Metric &\tt   detg\\
\tt    Sqrt Det of Metric      &\tt   sdetG\\
\hline
\tt    Volume                  &\tt   VOL\\
\hline
\tt    Basis                   &\tt   b'idim \\
\tt    Vector Basis            &\tt   e.idim \\
\hline
\tt    S-forms                 &\tt   S'a'b\\
\hline
\multicolumn{2}{|c|}{\tt Spinorial S-forms} \\
\tt    Undotted S-forms   &\tt    SU.AB\\
\tt    Dotted S-forms     &\tt    SD.AB\cc\\
\hline\end{tabular}

\subsection{\sf\slshape  Rotation Matrices}
\begin{tabular}{|l|l|}\hline
\tt    Frame Transformation      &\tt   L'a.b \\
\tt    Spinorial Transformation  &\tt   LS.A'B \\
\hline\end{tabular}

\subsection{\sf\slshape  Connection and related objects}
\begin{tabular}{|l|l|}\hline
\tt    Frame Connection     &\tt   omega'a.b\\
\tt    Holonomic Connection &\tt   GAMMA\^m\_n\\
\hline
\multicolumn{2}{|c|}{\tt Spinorial Connection}\\
\tt    Undotted Connection  &\tt   omegau.AB\\
\tt    Dotted Connection    &\tt   omegad.AB\cc\\
\hline
\tt    Riemann Frame Connection     &\tt   romega'a.b\\
\tt    Riemann Holonomic Connection &\tt   RGAMMA\^m\_n\\
\hline
\multicolumn{2}{|c|}{\tt Riemann Spinorial Connection}\\
\tt    Riemann Undotted Connection  &\tt   romegau.AB\\
\tt    Riemann Dotted Connection    &\tt   romegad.AB\cc\\
\hline
\tt    Connection Defect  &\tt    K'a.b\\
\hline\end{tabular}

\subsection{\sf\slshape  Torsion}
\begin{tabular}{|l|l|}\hline
\tt    Torsion    &\tt  THETA'a\\
\tt    Contorsion &\tt  KQ'a.b\\
\tt    Torsion Trace 1-form         &\tt   QQ\\
\tt    Antisymmetric Torsion 3-form &\tt  QQA\\
\hline
\multicolumn{2}{|c|}{\tt Spinorial Contorsion}\\
\tt    Undotted Contorsion   &\tt  KU.AB\\
\tt    Dotted Contorsion     &\tt  KD.AB\cc\\
\hline
\multicolumn{2}{|c|}{\tt    Torsion Spinors    }\\
\multicolumn{2}{|c|}{\tt    Torsion Components }\\
\tt    Torsion Trace               &\tt    QT'a\\
\tt    Torsion Pseudo Trace        &\tt    QP'a\\
\tt    Traceless Torsion Spinor    &\tt    QC.ABC.D\cc\\
\hline
\multicolumn{2}{|c|}{\tt    Torsion 2-forms}\\
\tt    Traceless Torsion 2-form     &\tt   THQC'a\\
\tt    Torsion Trace 2-form         &\tt   THQT'a\\
\tt    Antisymmetric Torsion 2-form &\tt   THQA'a\\
\hline
\multicolumn{2}{|c|}{\tt    Undotted Torsion 2-forms}\\
\tt    Undotted Torsion Trace 2-form         &\tt   THQTU'a\\
\tt    Undotted Antisymmetric Torsion 2-form &\tt   THQAU'a\\
\tt    Undotted Traceless Torsion 2-form     &\tt   THQCU'a\\
\hline\end{tabular}

\subsection{\sf\slshape  Nonmetricity}
\begin{tabular}{|l|l|}\hline
\tt    Nonmetricity        &\tt   N.a.b\\
\tt    Nonmetricity Defect &\tt   KN'a.b\\
\tt    Weyl Vector         &\tt   NNW\\
\tt    Nonmetricity Trace  &\tt   NNT\\
\hline
\multicolumn{2}{|c|}{\tt    Nonmetricity 1-forms}\\
\tt    Symmetric Nonmetricity 1-form     &\tt   NC.a.b\\
\tt    Antisymmetric Nonmetricity 1-form &\tt   NA.a.b\\
\tt    Nonmetricity Trace  1-form        &\tt   NT.a.b\\
\tt    Weyl Nonmetricity 1-form          &\tt   NW.a.b\\
\hline\end{tabular}

\subsection{\sf\slshape  Curvature}
\begin{tabular}{|l|l|}\hline
\tt    Curvature           &\tt   OMEGA'a.b\\
\hline
\multicolumn{2}{|c|}{\tt    Spinorial Curvature}\\
\tt    Undotted Curvature  &\tt   OMEGAU.AB\\
\tt    Dotted Curvature    &\tt   OMEGAD.AB\cc\\
\hline
\tt    Riemann Tensor      &\tt   RIM'a.b.c.d\\
\tt    Ricci Tensor        &\tt   RIC.a.b\\
\tt    A-Ricci Tensor      &\tt   RICA.a.b\\
\tt    S-Ricci Tensor      &\tt   RICS.a.b\\
\tt    Homothetic Curvature &\tt  OMEGAH\\
\tt    Einstein Tensor      &\tt  GT.a.b\\
\hline
\multicolumn{2}{|c|}{\tt    Curvature Spinors}\\
\multicolumn{2}{|c|}{\tt    Curvature Components}\\
\tt    Weyl Spinor                &\tt  RW.ABCD\\
\tt    Traceless Ricci Spinor     &\tt  RC.AB.CD\cc\\
\tt    Scalar Curvature           &\tt  RR\\
\tt    Ricanti Spinor             &\tt  RA.AB\\
\tt    Traceless Deviation Spinor &\tt  RB.AB.CD\cc\\
\tt    Scalar Deviation           &\tt  RD\\
\hline
\multicolumn{2}{|c|}{\tt Undotted Curvature 2-forms}\\
\tt    Undotted Weyl 2-form                &\tt  OMWU.AB \\
\tt    Undotted Traceless Ricci 2-form     &\tt  OMCU.AB \\
\tt    Undotted Scalar Curvature 2-form    &\tt  OMRU.AB \\
\tt    Undotted Ricanti 2-form             &\tt  OMAU.AB \\
\tt    Undotted Traceless Deviation 2-form &\tt  OMBU.AB \\
\tt    Undotted Scalar Deviation 2-form    &\tt  OMDU.AB \\
\hline
\multicolumn{2}{|c|}{\tt  Curvature 2-forms}\\
\tt    Weyl 2-form                     &\tt    OMW.a.b \\
\tt    Traceless Ricci 2-form          &\tt    OMC.a.b \\
\tt    Scalar Curvature 2-form         &\tt    OMR.a.b \\
\tt    Ricanti 2-form                  &\tt    OMA.a.b \\
\tt    Traceless Deviation 2-form      &\tt    OMB.a.b \\
\tt    Antisymmetric Curvature 2-form  &\tt    OMD.a.b \\
\tt    Homothetic Curvature 2-form     &\tt    OSH.a.b \\
\tt    Antisymmetric S-Ricci 2-form    &\tt  OSA.a.b  \\
\tt    Traceless S-Ricci 2-form        &\tt  OSC.a.b  \\
\tt    Antisymmetric S-Curvature 2-form &\tt  OSV.a.b  \\
\tt    Symmetric S-Curvature 2-form     &\tt  OSU.a.b  \\
\hline
\end{tabular}


\subsection{\sf\slshape  EM field}
\begin{tabular}{|l|l|}\hline
\tt    EM Potential    &\tt    A\\
\tt    Current 1-form  &\tt    J\\
\tt    EM Action       &\tt    EMACT\\
\tt    EM 2-form       &\tt    FF\\
\tt    EM Tensor       &\tt    FT.a.b\\
\hline
\multicolumn{2}{|c|}{\tt    Maxwell Equations}\\
\tt    First Maxwell Equation    &\tt    MWFq\\
\tt    Second Maxwell Equation   &\tt    MWSq\\
\hline
\tt    Continuity Equation       &\tt  COq\\
\tt    EM Energy-Momentum Tensor &\tt  TEM.a.b\\
\hline
\multicolumn{2}{|c|}{\tt    EM Scalars}\\
\tt    First EM Scalar         &\tt      SCF\\
\tt    Second EM Scalar        &\tt      SCS\\
\hline
\tt    Selfduality Equation    &\tt    SDq.AB\cc\\
\tt    Complex EM 2-form        &\tt   FFU\\
\tt    Complex Maxwell Equation &\tt   MWUq\\
\tt    Undotted EM Spinor       &\tt   FIU.AB\\
\tt    Complex EM Scalar        &\tt   SCU\\
\tt    EM Energy-Momentum Spinor &\tt  TEMS.AB.CD\cc\\
\hline\end{tabular}

\subsection{\sf\slshape  Scalar field}
\begin{tabular}{|l|l|}\hline
\tt    Scalar Equation       &\tt  SCq\\
\tt    Scalar Field          &\tt  FI\\
\tt    Scalar Action         &\tt  SACT\\
\tt    Minimal Scalar Action &\tt  SACTMIN\\
\tt    Minimal Scalar Energy-Momentum Tensor &\tt  TSCLMIN.a.b\\
\hline\end{tabular}


\subsection{\sf\slshape YM field}
\begin{tabular}{|l|l|}\hline
\tt    YM Potential         &\tt  AYM.i9\\
\tt    Structural Constants &\tt  SCONST.i9.j9.k9\\
\tt    YM Action            &\tt  YMACT\\
\tt    YM 2-form          &\tt  FFYM.i9\\
\tt    YM Tensor          &\tt   FTYM.i9.a.b\\
\hline
\multicolumn{2}{|c|}{\tt    YM Equations}\\
\tt    First YM Equation  &\tt   YMFq.i9\\
\tt    Second YM Equation &\tt   YMSq.i9\\
\hline
\tt    YM Energy-Momentum Tensor &\tt  TYM.a.b\\
\hline\end{tabular}

\subsection{\sf\slshape  Dirac field}
\begin{tabular}{|l|l|}\hline
\multicolumn{2}{|c|}{\tt    Dirac Spinor}\\
\tt    Phi Spinor   &\tt   PHI.A\\
\tt    Chi Spinor   &\tt   CHI.B\\
\hline
\tt    Dirac Action 4-form &\tt  DACT\\
\tt    Undotted Dirac Spin 3-Form &\tt  SPDIU.AB\\
\tt    Dirac Energy-Momentum Tensor &\tt  TDI.a.b\\
\hline
\multicolumn{2}{|c|}{\tt    Dirac Equation}\\
\tt    Phi Dirac Equation  &\tt   DPq.A\cc\\
\tt    Chi Dirac Equation  &\tt   DCq.A\cc\\
\hline\end{tabular}

\subsection{\sf\slshape  Geodesics}
\begin{tabular}{|l|l|}\hline
\tt    Geodesic Equation  &\tt   GEOq\^m\\
\hline\end{tabular}

\subsection{\sf\slshape  Null Congruence}
\begin{tabular}{|l|l|}\hline
\tt    Congruence                    &\tt  KV\\
\tt    Null Congruence Condition     &\tt  NCo\\
\tt    Geodesics Congruence Condition&\tt  GCo'a\\
\hline
\multicolumn{2}{|c|}{\tt    Optical Scalars}\\
\tt    Congruence Expansion          &\tt  thetaO\\
\tt    Congruence Squared Rotation   &\tt  omegaSQO\\
\tt    Congruence Squared Shear      &\tt  sigmaSQO\\
\hline\end{tabular}

\subsection{\sf\slshape  Kinematics}
\begin{tabular}{|l|l|}\hline
\tt    Velocity Vector  &\tt   UV\\
\tt    Velocity         &\tt   UU'a\\
\tt    Velocity Square  &\tt   USQ\\
\tt    Projector        &\tt   PR'a.b\\
\hline
\multicolumn{2}{|c|}{\tt    Kinematics}\\
\tt    Acceleration     &\tt   accU'a\\
\tt    Vorticity        &\tt   omegaU.a.b\\
\tt    Volume Expansion &\tt   thetaU\\
\tt    Shear            &\tt   sigmaU.a.b\\
\hline\end{tabular}

\subsection{\sf\slshape  Ideal and Spin Fluid}
\begin{tabular}{|l|l|}\hline
\tt    Pressure                           &\tt  PRES\\
\tt    Energy Density                     &\tt  ENER\\
\tt    Ideal Fluid Energy-Momentum Tensor &\tt  TIFL.a.b\\
\hline
\tt    Spin Fluid Energy-Momentum Tensor &\tt  TSFL.a.b \\
\tt    Spin Density                      &\tt  SPFLT.a.b \\
\tt    Spin Density 2-form               &\tt  SPFL \\
\tt    Undotted Fluid Spin 3-form        &\tt  SPFLU.AB \\
\tt    Frenkel Condition                 &\tt  FCo \\
\hline\end{tabular}

\subsection{\sf\slshape  Total Energy-Momentum and Spin}
\begin{tabular}{|l|l|}\hline
\tt    Total Energy-Momentum Tensor &\tt   TENMOM.a.b\\
\tt    Total Energy-Momentum Spinor &\tt   TENMOMS.AB.CD\cc\\
\tt    Total Energy-Momentum Trace  &\tt   TENMOMT\\
\tt    Total Undotted Spin 3-form   &\tt   SPINU.AB\\
\hline\end{tabular}

\subsection{\sf\slshape  Einstein Equations}
\begin{tabular}{|l|l|}\hline
\tt    Einstein Equation           &\tt   EEq.a.b\\
\hline
\multicolumn{2}{|c|}{\tt    Spinor Einstein Equations}\\
\tt    Traceless Einstein Equation &\tt   CEEq.AB.CD\cc\\
\tt    Trace of Einstein Equation  &\tt   TEEq\\
\hline\end{tabular}

\subsection{\sf\slshape Constants}
\begin{tabular}{|l|l|}\hline
\tt    A-Constants &\tt   ACONST.i2\\
\tt    L-Constants &\tt   LCONST.i6\\
\tt    M-Constants &\tt   MCONST.i3\\
\hline\end{tabular}

\subsection{\sf\slshape  Gravitational Equations}
\begin{tabular}{|l|l|}\hline
\tt    Action                      &\tt  LACT\\
\tt    Undotted Curvature Momentum &\tt  POMEGAU.AB\\
\tt    Torsion Momentum            &\tt  PTHETA'a\\
\hline
\multicolumn{2}{|c|}{\tt    Gravitational Equations}\\
\tt    Metric Equation             &\tt  METRq.a.b\\
\tt    Torsion Equation            &\tt  TORSq.AB\\
\hline\end{tabular}

\end{document}

%========  End of guide32.tex  ============================================%

